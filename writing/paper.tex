\documentclass{report}

\newcommand{\note}[1]{$\langle$ #1 $\rangle$}


\begin{document}

%%%%%%%%%%%%%%%%%%%%%%%%%%%%%%%%%%%%%%%%%%%%%%%%%%%%%%%%%%%%%%%%%%%%%%%%%%%%%%%%%
\begin{abstract}
\end{abstract}

%%%%%%%%%%%%%%%%%%%%%%%%%%%%%%%%%%%%%%%%%%%%%%%%%%%%%%%%%%%%%%%%%%%%%%%%%%%%%%%%%
\section{Introduction}

%%%%%%%%%%%%%%%%%%%%%%%%%%%%%%%%%%%%%%%%%%%%%%%%%%%%%%%%%%%%%%%%%%%%%%%%%%%%%%%%%
\section{Previous and Related Work}

\subsection{Breadth first Search}
\subsection{Uniform Cost Search}
\subsection{A* Search}
\subsection{Bootstrapping}
\subsection{Learning During Search}
\subsection{Probing Search}
\subsection{Planning with Preferred Operators}


%%%%%%%%%%%%%%%%%%%%%%%%%%%%%%%%%%%%%%%%%%%%%%%%%%%%%%%%%%%%%%%%%%%%%%%%%%%%%%%%%
\section{Proposed Algorithm}


\subsection{definitions}
\begin{description}
  \item[$\mathit{start}$] The initial state of the search \note{we aren't
    considering multiple initial configurations? -- JTT}
  \item[$d(n)$] an estimate of the number of actions laying between
    some node $n$ and a goal node.
  \item[$D(n)$] the number of transitions used to reach some node $n$
    from the initial state of the search.
  \item[$L(n)$] $L(n) = d(n) + D(n)$ is an estimate of the total
    length of a plan passing through $n$, using the current partial
    plan at $n$.
  \item[$h(n)$] an estimate of the cost of the actions laying between
    some node $n$ and a goal node.
  \item[$g(n)$] the cost incurred when reaching $n$ from the initial
    state by the current path.
  \item[$f(n)$] an estimate of the total cost of a plan passing
    through $n$, using the partial plan that currently results in $n$.
  \item[$\phi(n)$] A set of features (for learning) related to the
    search node $n$. \note{Note the distinction of node versus state
      here.  I'm not sure if we mean to use nodes versus states, and I
      can see arguments going both ways on this, but I'll argue for
      nodes -- the experience of the search itself is key to getting
      shit like this to work (I think), and the state is divorced from
      the experience -- JTT}
  \item[$\gamma$] The learned ranking function.
  \item[$c(n,m)$] The cost of transitioning between nodes $n$ and $m$.
  \item[$l(n,m)$] The number of transitions between nodes $n$ and $m$.
\end{description}

\subsection{Algorithm Description}

  open is initially the set containing the initial state.\\
  closed is initially the set containing the initial state, with $g(n) = 0$.\\

  \begin{enumerate}
  \item While we haven't found the goal
  \item select the {\em single best} candidate out of open
  \item perform breadth first search from this node for $N$ steps,
    using the closed list from the outer algorithm to replace nodes
    that are seen via a worse path with their better counterpart.
  \item After $N$ steps, use $\gamma$ to order the frontier of the
    inner breadth first search.
  \item Select the best node according to $\gamma$, and perform
    training, to learn $\gamma'$
  \item Start over at the second step, using the single best node
    according to $\gamma$. Replace $\gamma$ with $\gamma'$.
  \end{enumerate}

\subsubsection{Reservations about this approach}

\begin{enumerate}
  \item We should justify breadth first search as the right choice
    (as opposed to uniform cost search)
  \item How the hell are we going to set $N$, the look ahead
    parameter.
  \item $\phi$ is also problematic.  You have to balance having an
    informed feature set with something that's relatively cheap to
    compute.
  \item The learning step is difficult enough that I broke it's
    discussion out into a single section.
\end{enumerate}

\subsubsection{Accounting for other good ideas}
  How do we incorporate the idea of preferred operators, planning with
  probes, dovetailing over several algorithms that may potentially
  produce catastrophic failure, and so on into our algorithm?


%%%%%%%%%%%%%%%%%%%%%%%%%%%%%
\subsection{The Learning Step}

\begin{enumerate}
  \item Learn a function that predicts true $L$ or true $f$ from the
    starting node, and the single best element of the frontier.
  \item Use the path, from frontier node back to start, and all nodes
    along it, to learn a function that predicts true $L$ or true $f$.
  \item Learn a function that ranks the frontier of the inner search
    appropriately.
  \item Learn a function that ranks every layer of the inner search
    appropriately.  So, for steps 1..$N$, and nodes along the path
    from start to best along the frontier, learn a ranking which
    places each node on the path as the best among its peers.
\end{enumerate}

\note{1, 2, and 3 follow directly from what we want to be able to do
  -- rank the frontier appropriately so that we can pick the correct
  next starting state.  Where they differ is in what is learned, 1 and
  2 learn functions, 3 learns a ranking. 4 actually comes from the
  same idea as 2, we want to learn $X$ but are worried that we may
  have insufficient data to learn $X$.  What is the most data we can
  reasonably get to learn $X$? Well, in this case, it's learning from
  all of the frontiers generated by the local search step.}

%%%%%%%%%%%%%%%%%%%%%%%%%%%%%%%%%%%%%%%%%%%%%%%%%%%%%%%%%%%%%%%%%%%%%%%%%%%%%%%%%
\section{Experiments}

%%%%%%%%%%%%%%%%%%%%%%%%%%%%%%%%%%%%%%%%%%%%%%%%%%%%%%%%%%%%%%%%%%%%%%%%%%%%%%%%%
\section{Discussion}

%%%%%%%%%%%%%%%%%%%%%%%%%%%%%%%%%%%%%%%%%%%%%%%%%%%%%%%%%%%%%%%%%%%%%%%%%%%%%%%%%
\section{Conclusion}

\end{document}
